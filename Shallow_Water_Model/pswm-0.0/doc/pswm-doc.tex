% requires LaTeX (LaTeX2e)
\documentclass[12pt]{article}

% ****************** BASIC FORMATTING ************************

\setlength{\textheight}{9.0in}
\setlength{\textwidth}{6.5in}
\setlength{\topmargin}{0in}
\setlength{\headheight}{0in}
\setlength{\headsep}{0in}
%\setlength{\footskip}{0in}
\setlength{\oddsidemargin}{0in}
\setlength{\evensidemargin}{0in}
\setlength{\marginparsep}{0in}
\setlength{\marginparwidth}{0in}

% ****************** BIBLIOGRAPHY ************************

%\usepackage{AMS}
\bibliographystyle{AMS}

% ****************** GRAPHICS ************************

\usepackage[final]{graphicx} % works with latex, pdflatex--leave off extensions

% ****************** OTHER PACKAGES ************************

% HYPERREF:  This package creates hyperlinks to equations, sections, 
% references, etc.  It works with either LaTeX or pdflatex
% and the resulting links work with acrobat, xpdf, and xdvi
% PUT THIS EARLY--IT POLLUTES THE NAMESPACE WITH THINGS LIKE \v
%\usepackage{hyperref}

\usepackage{amssymb, amsmath, amsfonts}

% ****************** MORE FORMATTING ************************

%\setlength{\parindent}{0pt}
\setlength{\parskip}{12pt plus6pt minus3pt}
\setlength{\jot}{6pt} % extra space between equations in \eqnarray or \align

% commands to number equations by section, e.g., (2.3):
\renewcommand{\theequation}{\thesection.\arabic{equation}}
% NOTE:  you must also reset the equation counter "equation" at each section:
%\section{Title of Section}
%\setcounter{equation}{0}

% Another approach (which doesn't seem to work):
%%% The following set of commands will force the numbering of equations
%%% to restart automatically when entering a new section
%\makeatletter      % '@' is now a normail "letter" for TeX
%\@addtoreset{equation}{section}
%\makeatother       % '@' is restored as a "non-letter" character for TeX

% Theorem Environments
\newtheorem{theorem}{Theorem}[section]
%\newtheorem{lemma}[theorem]{Lemma}
\newtheorem{lemma}{Lemma}
\newtheorem{proposition}[theorem]{Proposition}
\newtheorem{corollary}[theorem]{Corollary}
\newenvironment{proof}{\par\noindent\textbf{Proof: }}{\hfill\qed\par}
\newcommand{\qed}{\fbox{\rule[-0.0ex]{0pt}{1.5ex}\hspace*{0.1em}}}

% macros for narrower captions:
% \mycaption{text} or \mycaption[width]{text}
\newcommand{\mycaption}[2][5.5in]{\parbox[t]{#1}{\caption{#2}}}
%\newcommand{\mycaption}[1]{\Mycaption{5.5in}{#1}}
%\newcommand{\Mycaption}[2]{\parbox[t]{#1}{\caption{#2}}}

% ****************** SYMBOLS ************************

\usepackage{amsfonts}

% standard mathematical symbols
\newcommand{\R}{\mathbb{R}}
\newcommand{\C}{\mathbb{C}}
\newcommand{\Z}{\mathbb{Z}}
% absolute value, norm, condition number, and inner product:
%     \abs{stuff}
%     \norm{stuff}  or \norm[type]{stuff}   ex:  \norm[\infty]{x}
%     \cond{stuff}  or \cond[type]{stuff}   ex:  \cond[2]{x}
%     \ip{arg}{arg} or \ip[type]{arg}{arg}  ex:  ex:  \ip[A]{f}{g}
\newcommand{\abs}[1]{\left|{#1}\right|}
\newcommand{\norm}[2][{}]{\ensuremath{\left\|{#2}\right\|_{#1}}}
\newcommand{\cond}[2][{}]{\ensuremath{\text{cond}\!\left({#2}\right)}_{#1}}
\newcommand{\ip}[3][{}]{\ensuremath{\left\langle{#2},{#3}\right\rangle_{#1}}}

% symbols for complex numbers
\newcommand{\cc}[1]{\bar{#1}}           % complex conjugate
\newcommand{\Cc}[1]{\overline{#1}}      % longer version of complex conjugate
\newcommand{\CC}[1]{{#1}^{*}}           % asterisk as alternate version
\renewcommand\Re{\real} \DeclareMathOperator{\real}{Re} % looks better than
\renewcommand\Im{\imag} \DeclareMathOperator{\imag}{Im} % standard LaTex

% symbols for use with sets and functions
\newcommand{\set}[1]{\left\{{#1}\right\}}  % use $\set{stuff}$ for a set
\newcommand{\suchthat}{:}                  % for use with sets
\newcommand{\from}{\colon}                 % for use with functions

% symbols for variables, operators, and equations
\newcommand{\vecv}{\mathbf{v}}
\newcommand{\vecV}{\mathbf{V}}
\newcommand{\veck}{\mathbf{k}}
\newcommand{\Fmom}{\mathbf{F}}
\newcommand{\Fmass}{F_h}
\newcommand{\Fphi}{F_\p}
\newcommand{\Fzeta}{F_{\zeta}}
\newcommand{\Fdelta}{F_{\delta}}
\newcommand{\solvec}{\psi}
\newcommand{\href}{\bar{h}}
\renewcommand{\v}{\mathbf{v}}
\newcommand{\p}{{p}}   % mass variable:  scaled deviation geopotential
\renewcommand{\P}{{P}} % corresponding forcing
\newcommand{\PV}{q}    % potential vorticity
\newcommand{\FPV}{F_q} % corresponding forcing
\newcommand{\opA}{\mathcal{A}}
\newcommand{\opB}{\mathcal{B}}
\newcommand{\opC}{\mathcal{C}}

% symbols for vectors and vector operations
\newcommand{\kbar}{\bar{k}}
\newcommand{\cross}{\times}
\newcommand{\del}{\nabla}
\newcommand{\laplacian}[1]{\del^2#1}
%\newcommand{\fluxdiv}[2]{\mathcal{F}\left(#1,#2\right)}
\newcommand{\fluxdiv}[2]{\del\cdot\left(#1\del#2\right)}
\newcommand{\jacobian}[2]{\mathcal{J}\left(#1,#2\right)}

% symbols for Fourier analysis
\newcommand{\fc}[1]{\widetilde{#1}} % Fourier (series) coefficient
\newcommand{\dfc}[1]{\widehat{#1}}  % discrete Fourier coefficient
\newcommand{\Fc}[1]{\dfc{#1}}       % alternate name

% symbols for discretization
\newcommand{\dt}{\Delta t}

% other symbols for this paper
\newcommand{\code}[1]{\texttt{#1}}
\newcommand{\pswm}{\code{pswm}}

% ****************** OTHER DEFINITIONS ************************

\newcommand{\units}[1]{\,\mbox{#1}}

% ****************** FRONT MATTER ************************

\title{PSWM: a Periodic Shallow-Water Model}

\author{Scott R. Fulton
\\[12pt]
\emph{Department of Mathematics and Computer Science}\\
\emph{Clarkson University, Potsdam, NY 13699--5815}
}

\date{Revised 19 July 2007}
%\date{Revised \today}

% ****************** START OF DOCUMENT ************************

\begin{document}
\maketitle

% ****************** ABSTRACT ************************

\begin{abstract}
\noindent
These notes summarize several mathematical details underlying the periodic
shallow water model \pswm, including the equation forms used for semi-implicit
time differencing and the details of the Fourier transforms used.
\end{abstract}

%\thispagestyle{empty}

% ****************** TEXT ************************

\pagebreak[3]
\section{Equation Forms\label{sec:equations}}
\setcounter{equation}{0}

The shallow-water equations may be formulated in various ways for
semi-implicit time differencing, based on various choices of the form of 
equations and which terms to treat implicitly.  A fairly comprehensive
overview is given in~\cite{Fulton07}; here we consider only the details of the
equations in the various forms used in the periodic shallow water model \pswm.
In each case we write the time-continuous equations in the split form
\begin{equation}
  \solvec'(t) - \opA(\solvec(t)) = \opB(\solvec(t)) ,
\label{eq:ODE}
\end{equation}
where $\solvec(t)$ represents the solution variables,
$\opA(\solvec(t))$ represents the terms to be treated implicitly, and
$\opB(\solvec(t))$ represents the terms to be treated explicitly.
When discretized by an $m$-step semi-implicit scheme using solution values
$\solvec^n\approx\solvec(t_n)$ with $t_n=n\dt$, the corresponding implicit
system to be solved for $\solvec^{n+1}$ takes the form
\begin{equation}
  \solvec^{n+1}-\tau\opA(\solvec^{n+1}) = \opC^{n},
\label{eq:implicit}
\end{equation}
where $\tau$ is a multiple of the time step and $\opC^{n}$ is a linear
combination of values of $\solvec$, $\opA(\solvec)$, $\opB(\solvec)$ at
times~$t_n$, $t_{n-1}$, \dots, $t_{n-m+1}$.  In the model $\opC^{n}$ is
generated by the package \verb+sitpack+ based on the explicit and implicit
schemes chosen; the details do not concern us here.  In the sections that
follow we express the model equations in a time-continuous split form
corresponding to~\eqref{eq:ODE} and a time-discrete implicit system
corresponding to~\eqref{eq:implicit}, and show how to solve that system.  
We do this below for 
the momentum form, %(sec.~\ref{sec:momentum}),
the vorticity/divergence form, %(sec.~\ref{sec:vd}),
and the potential vorticity/divergence form. %(sec.~\ref{sec:pvd}).

\pagebreak[2]
\subsection{Momentum Form\label{sec:momentum}}

We can write the momentum and mass continuity equations of the shallow-water
system in the form
\begin{align}
  \frac{\partial \vecv}{\partial t} + \vecv\cdot\del\vecv 
    + f\veck\cross\vecv + g\del h &= \Fmom, 
\label{eq:momentum-h}
\\
  \frac{\partial h}{\partial t} + \del\cdot(h\vecv) &= \Fmass,
\label{eq:continuity-h}
\end{align}
where $\vecv$ is the (horizontal) vector velocity, $h$ is the free surface
height, $f$ is the Coriolis parameter, $g$ is the gravitational constant,
$\veck$ is the vertical unit vector, and the operator $\del$ is restricted to
the horizontal.  The source terms $\Fmom$ for momentum and $\Fmass$ for mass
are regarded as known functions of $t$ (and possibly of $\vecv$ and $h$).

In this model we identify the gravity-wave terms using a splitting based on a
constant reference depth $\href$.  It is convenient to work with the scaled
deviation geopotential $\p := g(h-\href)/c$, where $c:=(g\href)^{1/2}$, in
place of the depth.\footnote{This eliminates $g$ from the problem and gives
variables which have the same units.}  With this change, 
\eqref{eq:momentum-h} and \eqref{eq:continuity-h} become
\begin{align}
  \frac{\partial \vecv}{\partial t} + \vecv\cdot\del\vecv 
    + f\veck\cross\vecv + c\del \p &= \Fmom, 
\label{eq:momentum}
\\
  \frac{\partial\p}{\partial t} + c\del\cdot\vecv + \del\cdot(\p\vecv) 
      &= \Fphi ,
\label{eq:continuity}
\end{align}
where $\Fphi := g\Fmass/c$.
It should be noted that the advective form \eqref{eq:momentum} is equivalent
to the rotational form
\begin{equation}
  \frac{\partial \vecv}{\partial t} + (f+\zeta)\veck\cross\vecv 
   + \del(c\p + K) = \Fmom, 
\label{eq:rotational}
\end{equation}
where $\zeta = \veck\cdot\del\cross\vecv$ is the relative vorticity and
$K=\frac12\vecv\cdot\vecv$ is the specific kinetic energy.  The differences
between these two forms affect only the nonlinear terms---which will be
treated explicitly---and thus do not impact the implicit system to be solved.

To use a semi-implicit time discretization, we put the terms to be treated 
implicitly (height gradient and linear part of the divergence term) on the
left and those to be treated explicitly (including the Coriolis terms for
simplicity) on the right [cf.~\eqref{eq:ODE}].  This gives the split form
\begin{align}
   \frac{\partial \vecv}{\partial t} + c\del \p &= 
      \Fmom - \vecv\cdot\del\vecv - f\veck\cross\vecv 
\label{eq:momentum:split}
\\
   \frac{\partial\p}{\partial t} + c\del\cdot\vecv 
      &= \Fphi - \del\cdot(\p\vecv) .
\label{eq:continuity:split}
\end{align}
With any semi-implicit scheme, the implicit system [cf.~\eqref{eq:implicit}]
generated from \eqref{eq:momentum:split}--\eqref{eq:continuity:split} is
\begin{align}
   \vecv + \tau c\del \p &= \vecV ,
\label{eq:momentum:imp}
\\
   \p + \tau c\del\cdot\vecv &= \P ,
\label{eq:continuity:imp}
\end{align}
where $\tau$ is a multiple of the time step $\dt$, $\vecv$ and $\p$ now
represent the variables \emph{at the new time level} (i.e., $\solvec^{n+1}$
with the superscript dropped for simplicity), and $\vecV$ and $\P$ are
generated (by \verb+sitpack+) from the right-hand side of
\eqref{eq:momentum:split}--\eqref{eq:continuity:split} and values of $\vecv$
and $\p$ at the previous time level(s).  To solve this system, we substitute
for $\vecv$ from \eqref{eq:momentum:imp} in \eqref{eq:continuity:imp} to
obtain the modified Helmholtz equation
\begin{equation}
  \p - \tau^2 c^2\del^2 \p = G := \P - \tau c\del\cdot\vecV.
\label{eq:Helmholtz}
\end{equation}
Once this is solved for $\p$, the corresponding $\vecv$ can be obtained from
\eqref{eq:momentum:imp}.  If the solution of \eqref{eq:Helmholtz} is only
approximate, then $\p$ should be recomputed from $\vecv$ via
\eqref{eq:continuity:imp} to ensure exact mass continuity in the discretized
system.

\pagebreak[2]
\subsection{Vorticity/divergence form\label{sec:vd}}

The momentum form \eqref{eq:momentum}--\eqref{eq:continuity} treated above has
the disadvantage that the components of the velocity $\vecv$ are not true
scalars (i.e., their values depend on the coordinate system chosen).  An
alternate approach uses vorticity and divergence instead.  To do so, we take
the dot product of $\veck$ with the curl of the momentum equation in the form
\eqref{eq:rotational} to obtain the vorticity equation 
\begin{equation}
   \frac{\partial \zeta}{\partial t} + \del\cdot(\eta\vecv) =
      \Fzeta := \veck\cdot\del\cross\Fmom ,
\label{eq:vorticity}
\end{equation}
where $\eta:=f+\zeta$ is the absolute vorticity.  Likewise, taking the
divergence of \eqref{eq:rotational} gives the divergence equation
\begin{equation}
   \frac{\partial \delta}{\partial t} - \veck\cdot\del\cross(\eta\vecv) 
      + \del^2 \left(c\p + K\right) = \Fdelta := \del\cdot\Fmom ,
\label{eq:divergence}
\end{equation}
where $\delta = \del\cdot\vecv$ is the divergence.  Combining these equations
with \eqref{eq:continuity} gives a system for predicting $\zeta$, $\delta$,
and $h$.  To close this system we introduce a velocity potential $\chi$ and
streamfunction $\psi$ satisfying 
\begin{equation}
   \vecv = \del\chi + \veck\cross\del\psi ,
\label{eq:wind}
\end{equation}
which implies that
\begin{equation}
   \del^2\psi = \zeta, \qquad \del^2\chi = \delta .
\label{eq:Poisson}
\end{equation}

Coupling \eqref{eq:vorticity} and \eqref{eq:divergence} with the continuity
equation \eqref{eq:continuity} and putting the terms to be treated implicitly
on the left and those to be treated explicitly 
on the right gives the split form
[cf.~\eqref{eq:ODE}]
\begin{alignat}{2}
   &\frac{\partial \zeta}{\partial t} &&= \Fzeta - \del\cdot(\eta\vecv) ,
\label{eq:vorticity:split}
\\
   &\frac{\partial \delta}{\partial t} + c\del^2 \p &&= 
      \Fdelta + \veck\cdot\del\cross(\eta\vecv) - \del^2 K ,
\label{eq:divergence:split}
\\
   &\frac{\partial\p}{\partial t} + c\delta &&= \Fphi - \del\cdot(\p\vecv) .
\label{eq2:continuity:split}
\end{alignat}
If desired, the right-hand side of this system may be written in terms of 
$\chi$ and $\psi$ instead of~$\vecv$ using the identities
\begin{equation}
   \del\cdot(\alpha\vecv) = 
      \del\cdot(\alpha\del\chi) - \jacobian{\alpha}{\psi}
\label{eq:identity1}
\end{equation}
and
\begin{equation}
   \veck\cdot\del\cross(\alpha\vecv) = 
      \del\cdot(\alpha\del\psi) + \jacobian{\alpha}{\chi} ,
\label{eq:identity2}
\end{equation}
where $\jacobian{\alpha}{\beta}=\veck\cdot(\del\alpha\cross\del\beta)$ is the
Jacobian operator and $\alpha$ and $\beta$ represent any scalars. 
With these substitutions, 
\eqref{eq:vorticity:split}--\eqref{eq2:continuity:split} take the form
\begin{alignat}{2}
   &\frac{\partial \zeta}{\partial t} &&= 
      \Fzeta - \del\cdot(\eta\del\chi) + \jacobian{\eta}{\psi},
\label{eq:vorticity:split:psichi}
\\
   &\frac{\partial \delta}{\partial t} + c\del^2 \p &&= 
      \Fdelta + \del\cdot(\eta\del\psi) + \jacobian{\eta}{\chi} - \del^2 K ,
\label{eq:divergence:split:psichi}
\\
   &\frac{\partial\p}{\partial t} + c\delta &&= 
      \Fphi - \del\cdot(\p\del\chi) + \jacobian{\p}{\psi} .
\label{eq2:continuity:split:psichi}
\end{alignat}
In this way, $\vecv$ is eliminated, and if we express the kinetic energy as
\begin{equation}
   K = \frac12\left[ \fluxdiv{\chi}{\chi} - \chi\laplacian\chi
                   + \fluxdiv{\psi}{\psi} - \psi\laplacian\psi \right]
      + \jacobian{\psi}{\chi}.
\label{eq:K:psichi}
\end{equation}
then the only operators needed are the flux divergence, Jacobian, and
Laplacian.  However, this form is less efficient for the spectral model, since
it requires more terms to be transformed than the form
\eqref{eq:vorticity:split}--\eqref{eq2:continuity:split}.

With a semi-implicit time discretization, the implicit system generated from
\eqref{eq:vorticity:split}--\eqref{eq2:continuity:split} or
\eqref{eq:vorticity:split:psichi}--\eqref{eq2:continuity:split:psichi} is
\begin{alignat}{2}
   &\zeta &&= Z ,
\label{eq:vorticity:imp}
\\
   &\delta + \tau c\del^2 \p &&= D ,
\label{eq:divergence:imp}
\\
   &\p + \tau c\delta &&= \P ,
\label{eq2:continuity:imp}
\end{alignat}
where $\zeta$, $\delta$, and $\p$ now represent the variables at the new
time level and $Z$, $D$, and $\P$ are generated (via \verb+sitpack+) from
the right-hand side of
\eqref{eq:vorticity:split}--\eqref{eq2:continuity:split} and values of
$\zeta$, $\delta$, and $\p$ at the previous time level(s).  
To solve this system, we eliminate $\delta$ between 
\eqref{eq:divergence:imp} and \eqref{eq2:continuity:imp} to
obtain the modified Helmholtz equation
\begin{equation}
  \p - \tau^2 c^2\del^2 \p = G := \P - \tau c D
\label{eq2:Helmholtz}
\end{equation}
[cf.~\eqref{eq:Helmholtz}].
Once this is solved for $\p$, the corresponding $\delta$ can be obtained
from \eqref{eq2:continuity:imp}, $\zeta$ is given directly by
\eqref{eq:vorticity:imp}, and $\psi$ and $\chi$ are obtained by solving
\eqref{eq:Poisson}.  If needed, the corresponding $\vecv$ can be computed from
\eqref{eq:wind}.

\pagebreak[2]
\subsection{Potential vorticity/divergence form\label{sec:pvd}}

By combining the vorticity equation \eqref{eq:vorticity} with the continuity
equation \eqref{eq:continuity} we can obtain the potential vorticity equation
\begin{equation}
   \frac{\partial \PV}{\partial t} + \vecv\cdot\del\PV = \FPV ,
\label{eq:PV}
\end{equation}
where
\begin{equation}
   \PV := \frac{h}{\href}\eta = \frac{f+\zeta}{1+\p/c}
\label{def:PV}
\end{equation}
is the potential vorticity and
\begin{equation}
   \FPV := \frac{(1+\p/c)\Fzeta - \eta\Fmass/c}{(1+p/c)^2} ,
\label{def:FPV}
\end{equation}
is the corresponding forcing.  Coupling \eqref{eq:PV} with the divergence
equation \eqref{eq:divergence} and the continuity equation
\eqref{eq:continuity} and putting the terms to be treated implicitly on the
left and those to be treated explicitly on the right gives the split form
[cf.~\eqref{eq:ODE}]
\begin{alignat}{2}
   &\frac{\partial\PV}{\partial t} &&= \FPV - \vecv\cdot\del\PV ,
\label{eq:PV:split}
\\
   &\frac{\partial \delta}{\partial t} + c\del^2 \p &&= 
      \Fdelta + \veck\cdot\del\cross(\eta\vecv) - \del^2 K ,
\label{eq3:divergence:split}
\\
   &\frac{\partial\p}{\partial t} + c\delta &&= \Fphi - \del\cdot(\p\vecv) .
\label{eq3:continuity:split}
\end{alignat}
As before, $\vecv$ may be replaced in terms of $\psi$ and $\chi$ on the
right-hand side of this system using \eqref{eq:identity1} and
\eqref{eq:identity2}.

With a semi-implicit time discretization, the implicit system generated from
\eqref{eq:PV:split}--\eqref{eq3:continuity:split} is
\begin{alignat}{2}
   &\PV &&= Q ,
\label{eq:PV:imp}
\\
   &\delta + \tau c\del^2 \p &&= D ,
\label{eq3:divergence:imp}
\\
   &\p + \tau c\delta &&= \P ,
\label{eq3:continuity:imp}
\end{alignat}
where $\PV$, $\delta$, and $\p$ now represent the variables at the new
time level and $Q$, $D$, and $\P$ are generated (via \verb+sitpack+) from
the right-hand side of
\eqref{eq:PV:split}--\eqref{eq3:continuity:split} and values of
$\PV$, $\delta$, and $\p$ at the previous time level(s).  
To solve this system, we eliminate $\delta$ between 
\eqref{eq3:divergence:imp} and \eqref{eq3:continuity:imp} to
obtain the modified Helmholtz equation
\begin{equation}
  \p - \tau^2 c^2\del^2 \p = G := \P - \tau c D
\label{eq3:Helmholtz}
\end{equation}
[cf.~\eqref{eq:Helmholtz}].
Once this is solved for $\p$, the corresponding $\delta$ can be obtained from
\eqref{eq3:continuity:imp}, $\PV$ is given directly by \eqref{eq:PV:imp},
$\eta$ is obtained from \eqref{def:PV}, and $\psi$ and $\chi$ are obtained by
solving \eqref{eq:Poisson}.  If needed, the corresponding $\vecv$ can be
computed from \eqref{eq:wind}.

\pagebreak[2]
\subsection{Initialization\label{sec:initialization}}

To initialize any of the above forms of the model we can specify the predicted
variables, i.e., $u$, $v$, and $p$ for the momentum form, 
$\zeta$, $\delta$, and $p$ for the vorticity/divergence form, or
$\PV$, $\delta$, and $p$ for the potential vorticity/divergence form.
However, this will in general include a significant gravity-inertia wave
component in the initial fields.  A better approach is to use the nonlinear
balance equation as follows.

Setting $\partial\delta/\partial t=0$ and $\Fdelta=0$ in the divergence
equation \eqref{eq:divergence} and using Cartesian coordinates $(x,y)$ we
obtain
\begin{equation}
   - (\eta v)_x + (\eta u)_y + \del^2 \left(c\p + K\right) = 0 ,
\label{eq:div:1}
\end{equation}
where $u$ and $v$ are the velocity components in the $x$ and $y$ directions
and the subscripts denote partial derivatives.
Assuming the flow is purely rotational (nondivergent), then \eqref{eq:wind}
and \eqref{eq:K:psichi} reduce to
\begin{equation}
   u = -\psi_y,
\qquad
   v =  \psi_x,
\qquad
   K = \tfrac12\left( \psi_x^2 + \psi_y^2 \right).
\label{eq:wind:K:nondivergent}
\end{equation}
With these substitutions (and a lot of algebra), \eqref{eq:div:1} reduces to
the nonlinear balance equation
\begin{equation}
   f\laplacian\psi + 2\left(\psi_{xx}\psi_{yy} - \psi_{xy}^2 \right)
   = c\laplacian{p}.
\label{eq:NLB}
\end{equation}
If we specify the relative vorticity $\zeta$, the corresponding streamfunction
$\psi$ can be obtained by solving
\begin{equation}
   \laplacian\psi = \zeta,
\label{eq:psi:zeta}
\end{equation}
and then the corresponding height field $p$ (in nonlinear balance) can be
obtained by solving \eqref{eq:NLB}.  Note that with $\psi$ known this equation
is linear, and thus easy to solve.

\pagebreak[4]
\section{Fourier Transforms and Spectral-Space Operations\label{sec:Fourier}}
\setcounter{equation}{0}

The periodic shallow water model \pswm\ is formulated using a Fourier spectral
representation  on a two-dimensional rectangular domain which is periodic in
both directions.  In principle this is easy:  the Fourier modes are
eigenfunctions of the derivative and Laplacian operators, so these operations
can be computed (and inverted) easily in spectral space.  In practice, there
are some subtleties in the discrete version of this representation which must 
be taken into account.  In this section we first explain discrete Fourier 
representations in detail in one dimension, and then document how the
corresponding two-dimensional representations are used in the model.

\pagebreak[2]
\subsection{Fourier representations in one dimension}

\newcommand{\period}{L}
Consider a function $f\from\R\to\C$ which is $\period$-periodic, i.e.,
$f(x+\period)=f(x)$ for all $x\in\R$.  Such a function has the Fourier series
representation\footnote{The existence and convergence of this expansion depend
on the smoothness of $f$.  For example, if $f$ is continuously differentiable,
then the series exists and converges uniformly.  Here we ignore such
questions.}
\begin{equation}
   f(x) = \sum_{k=-\infty}^{\infty} \fc{f}_k e^{2\pi ikx/\period},
\qquad x\in\R,
\label{eq:FS}
\end{equation}
where
\begin{equation}
   \fc{f}_k = \frac{1}{\period}\int_{0}^{\period} f(x) 
              e^{-2\pi ikx/\period} \,dx ,
\qquad k\in\Z.
\label{eq:FC}
\end{equation}
Note from \eqref{eq:FC} that if $f$ is real then $\fc{f}_{-k} = \CC{\fc{f}}_k$
(where the asterisk represents the complex conjugate); in particular,
$\fc{f}_0$ is real.  We represent such a function on the computer by the
discrete values $f_j:=f(j\period/N)$ for integer values of~$j$, where $N$ is
some positive integer.  Since $f_{j+N}=f_j$ by periodicity, only $N$ values
are needed; normally we will use $j=0,\dots,N-1$.  Clearly this is not enough
information to determine the Fourier coefficients $\fc{f}_k$ for all $k\in\Z$.
Here we first derive the Discrete Fourier Transform pair, and then show how it
can be used to represent functions which are band-limited, i.e., exactly
represented by a \emph{truncated} Fourier series.  We then examine how such
Fourier representations can be used to compute derivatives and nonlinear
terms.

\pagebreak[1]
\subsubsection{Discrete Fourier transforms}

The key to discrete Fourier transforms is the following discrete orthogonality
property:

\begin{lemma}\label{lemma:orthogonality}
If $N$ is a positive integer and $j,k\in\Z$, then
\begin{equation}
   \frac{1}{N} \sum_{k=0}^{N-1} e^{2\pi ijk/N} = \begin{cases}
      1, & \text{$j\equiv 0\pmod{N}$}, \\
      0, & \text{otherwise}.
   \end{cases}
\label{eq:orthogonality}
\end{equation}
\end{lemma}
\begin{proof}
Setting $z=\exp(2\pi ij/N)$ we can sum the finite geometric series
\begin{equation}
   \sum_{k=0}^{N-1} e^{2\pi ijk/N} 
               = \sum_{k=0}^{N-1} z^{k} 
               = \begin{cases} N,    & \text{$z=1$},\\
                 \dfrac{1-z^N}{1-z}, & \text{$z\ne1$}. \end{cases}
\end{equation}
Since $z=1$ if and only if $j\equiv0\pmod{N}$ and $z^N = \exp(2\pi ij)=1$ for
any $j\in\Z$, this reduces to \eqref{eq:orthogonality}.
\end{proof}
This orthogonality property leads directly to the Discrete Fourier
Transform pair:
\begin{lemma}\label{lemma:DFT}
Two sequences $f_0,f_1,\dots,f_{N-1}$ and
$\dfc{f}_0,\dfc{f}_1,\dots,\dfc{f}_{N-1}$ of $N>0$ complex numbers are related
by the Discrete Fourier Transform (DFT)
\begin{equation}
   \dfc{f}_k = \frac{1}{N} \sum_{j=0}^{N-1} f_j e^{-2\pi ijk/N},
\qquad k=0,\dots,N-1 
\label{eq:DFT}
\end{equation}
if and only if they are related by the Inverse Discrete Fourier Transform 
(IDFT)
\begin{equation}
   f_j = \sum_{k=0}^{N-1} \dfc{f}_k e^{2\pi ijk/N},
\qquad j=0,\dots,N-1 .
\label{eq:IDFT}
\end{equation}
\end{lemma}
\begin{proof}
Direct calculation using \eqref{eq:DFT} with $j$ replaced by $l$ shows that
for $j=0,\dots,N-1$,
\begin{equation}
   \sum_{k=0}^{N-1} \dfc{f}_k e^{2\pi ijk/N}
   = \sum_{k=0}^{N-1} \left[
      \frac{1}{N} \sum_{l=0}^{N-1} f_l e^{-2\pi ilk/N}\right] e^{2\pi
ijk/N}
   = \sum_{l=0}^{N-1} \left[
      \frac{1}{N} \sum_{k=0}^{N-1} e^{2\pi i(j-l)k/N}\right] f_l = f_j,
\end{equation}
where the last step follows from Lemma~\ref{lemma:orthogonality}.
The converse follows by a similar argument.
\end{proof}
Both sequences $f_j$ and $\dfc{f}_k$ may be extended by periodicity to be
$N$-periodic.  Also, if the sequence $f_j$ is real then the sequence
$\dfc{f}_k$ is ``half-complex'', i.e., $\dfc{f}_{N-k} = \CC{\dfc{f}}_k$; in
particular, $\dfc{f}_0$ is real, as is $\dfc{f}_{N/2}$ if $N$ is even.  The
Fast Fourier Transform (FFT) is an algorithm for computing the DFT or IDFT in
$O\big(N\log(N)\big)$ operations; it normally uses a transform length $N$
which is either a power of~2 or an even number with many small prime factors.

\pagebreak[1]
\subsubsection{Representing functions by the DFT}

According to Lemma~\ref{lemma:DFT}, the DFT pair
\eqref{eq:IDFT}--\eqref{eq:DFT} relates two sequences of numbers, namely,
$f_0$, \dots, $f_{N-1}$ and $\dfc{f}_0$, \dots, $\dfc{f}_{N-1}$:  knowing
either sequence is equivalent to knowing the other.  To use this
representation in solving differential equations, we must relate this data
(i.e., either sequence of numbers) to a function $f(x)$ which we can
differentiate.  In particular, we want to relate the DFT coefficients
$\dfc{f}_k$ to the Fourier coefficients $\fc{f}_k$ of a function $f$ so we can
compute operations (such as derivatives) on $f$ in spectral space (i.e., in
terms of its Fourier coefficients).
Since there are only finitely many DFT coefficients, we can at best
expect them to determine a function with finitely many nonzero Fourier
coefficients, i.e., a \emph{band-limited} function of the form
\begin{equation}
   f(x) = \sum_{k=-M}^{M} \fc{f}_k e^{2\pi ikx/\period},
\qquad x\in\R.
\label{eq:FS:truncated}
\end{equation}
We refer to the integer $M$ here as the \emph{spectral truncation}; it
is the maximum index for which $\fc{f}_{k}$ or $\fc{f}_{-k}$ may be nonzero.
When $N>2M$ this relationship is as shown in Fig.~\ref{fig:coefficients}.
We can understand this relationship in two ways as follows.

\begin{figure}
\begin{center}
\setlength{\unitlength}{38pt}
\begin{picture}(11,4.5)(-3,-1)
%\thicklines
%\thinlines
\newcommand{\V}{3.2}   % vertical location of values
\renewcommand{\C}{1.6} % vertical location of coefficients
\newcommand{\A}{0.8}   % vertical location of arrows (lower)
\newcommand{\B}{2.4}   % vertical location of arrows (upper)
\put(-3,-0.8){\vector(0,1){0.4}}
\put(-3,-0.8){\line(1,0){9}}
\put(-3,0){\makebox(0,0)[c]{$\fc{f}_{-M}$}}
\put(-2,0){\makebox(0,0)[c]{$\dots$}}
\put(-1, 0){\makebox(0,0)[c]{$\fc{f}_{-1}$}}
\put(-1,-1){\vector(0,1){0.6}}
\put(-1,-1){\line(1,0){9}}
\put( 0, 0){\makebox(0,0)[c]{$\fc{f}_{ 0}$}}
\put( 0,\A){\makebox(0,0)[c]{$\updownarrow$}}
\put( 0,\C){\makebox(0,0)[c]{$\Fc{f}_{ 0}$}}
\put( 0,\C){\makebox(0,0)[r]{DFT coefficients:\qquad}}
\put( 0,\B){\makebox(0,0)[c]{$\updownarrow$}}
\put( 0,\V){\makebox(0,0)[c]{${f}_{ 0}$}}
\put( 0,\V){\makebox(0,0)[r]{Function values:\qquad}}
\put( 1, 0){\makebox(0,0)[c]{$\dots$}}
\put( 1,\C){\makebox(0,0)[c]{$\dots$}}
\put( 1,\V){\makebox(0,0)[c]{$\dots$}}
\put( 2, 0){\makebox(0,0)[c]{$\fc{f}_{ M}$}}
\put( 2,\A){\makebox(0,0)[c]{$\updownarrow$}}
\put( 2,\C){\makebox(0,0)[c]{$\Fc{f}_{ M}$}}
\put( 2,\B){\makebox(0,0)[c]{$\updownarrow$}}
\put( 2,\V){\makebox(0,0)[c]{${f}_{ M}$}}
\put( 3, 0){\makebox(0,0)[c]{$0$}}
\put( 3,\A){\makebox(0,0)[c]{$\updownarrow$}}
\put( 3,\C){\makebox(0,0)[c]{$\Fc{f}_{M+1}$}}
\put( 3,\B){\makebox(0,0)[c]{$\updownarrow$}}
\put( 3,\V){\makebox(0,0)[c]{${f}_{M+1}$}}
\put( 4, 0){\makebox(0,0)[c]{$\dots$}}
\put( 4,\C){\makebox(0,0)[c]{$\dots$}}
\put( 4,\V){\makebox(0,0)[c]{$\dots$}}
\put( 5, 0){\makebox(0,0)[c]{$0$}}
\put( 5,\A){\makebox(0,0)[c]{$\updownarrow$}}
\put( 5,\C){\makebox(0,0)[c]{$\Fc{f}_{N-M-1}$}}
\put( 5,\B){\makebox(0,0)[c]{$\updownarrow$}}
\put( 5,\V){\makebox(0,0)[c]{${f}_{N-M-1}$}}
\put( 6, 0){\makebox(0,0)[c]{$\fc{f}_{-M}$}}
\put( 6,-0.8){\vector(0,1){0.4}}
\put( 6,\A){\makebox(0,0)[c]{$\updownarrow$}}
\put( 6,\C){\makebox(0,0)[c]{$\Fc{f}_{N-M}$}}
\put( 6,\B){\makebox(0,0)[c]{$\updownarrow$}}
\put( 6,\V){\makebox(0,0)[c]{${f}_{N-M}$}}
\put( 7, 0){\makebox(0,0)[c]{$\dots$}}
\put( 7,\C){\makebox(0,0)[c]{$\dots$}}
\put( 7,\V){\makebox(0,0)[c]{$\dots$}}
\put( 8, 0){\makebox(0,0)[c]{$\fc{f}_{-1}$}}
\put( 8,-1){\vector(0,1){0.6}}
\put( 8,\A){\makebox(0,0)[c]{$\updownarrow$}}
\put( 8,\C){\makebox(0,0)[c]{$\Fc{f}_{N-1}$}}
\put( 8,\B){\makebox(0,0)[c]{$\updownarrow$}}
\put( 8,\V){\makebox(0,0)[c]{${f}_{N-1}$}}
\end{picture}
\caption{\label{fig:coefficients}
Relation between the coefficients of a DFT and a truncated Fourier series.}
\end{center}
\end{figure}

First, suppose the function $f$ is given as in \eqref{eq:FS:truncated}.
Evaluating this at the points $x_j=j\period/N$ gives the discrete values
\begin{equation}
   f_j = f(x_j) = \sum_{k=-M}^{M} \fc{f}_k e^{2\pi ijk/N},
\qquad j=0,\dots,N-1 .
\label{eq:FS:j}
\end{equation}
If $N>2M$ we can define the sequence $\dfc{f}_0,\dots,\dfc{f}_{N-1}$ by
\begin{equation}
   \dfc{f}_k := \begin{cases} 
      \fc{f}_k,     & \text{$k=0,\dots,M$}, \\
      0,            & \text{$k=M+1,\dots,N-M-1$}, \\
      \fc{f}_{k-N}, & \text{$k=N-M,\dots,N-1$}
   \end{cases}
\label{def:dfc:1}
\end{equation}
(see Fig.~\ref{fig:coefficients}).
Then \eqref{eq:FS:j} reduces to the IDFT \eqref{eq:IDFT}, so $\dfc{f}_k$ is
given by the DFT \eqref{eq:DFT}.  Thus, for the band-limited function
\eqref{eq:FS:truncated} the DFT coefficients $\dfc{f}_k$ are identical to the
Fourier series coefficients $\fc{f}_k$ (modulo~$N$) provided that $N>2M$. 
Mote that if we choose $N=2M+1$ then both the DFT and Fourier series
representations of $f$ use $N=2M+1$ degrees of freedom, and none of the 
coefficients \eqref{def:dfc:1} are automatically zero; if $N>2M+1$ then there
are ``extra'' terms in the DFT corresponding to Fourier coefficients which
vanish.

Conversely, suppose we know either of the two $N$-periodic sequences
$f_0,\dots,f_{N-1}$ or $\dfc{f}_0,\dots,\dfc{f}_{N-1}$ which are related by
the DFT pair.  If $N$ is \emph{odd}, this data uniquely determines a function
$f$ which satisfies $f(x_j)=f_j$ as follows:  setting $M=(N-1)/2$, we invert
\eqref{def:dfc:1} to yield $\fc{f}_k = \dfc{f}_k$ for $\abs{k}\le M$.  The
corresponding band-limited function $f$ given by \eqref{eq:FS:truncated} then
satisfies $f(x_j)=f_j$ for $j=0,\dots,N-1$.  Thus, if $N$ is odd we can
interpret the DFT coefficients $\dfc{f}_0,\dots,\dfc{f}_{N-1}$ directly as the
Fourier series coefficients $\fc{f}_k$ (modulo~$N$) of the band-limited
function given by \eqref{eq:FS:truncated}.

However, it is awkward to use $N$ odd in discrete Fourier representations,
since the FFT codes generally require $N$ to be even.  What happens if we try
to relate the sequence $f_0,\dots,f_{N-1}$ (or $\dfc{f}_0,\dots,\dfc{f}_{N-1}$
via the DFT) to a band-limited function $f$ of the form
\eqref{eq:FS:truncated} when $N$ is \emph{even}?  If we set $M=N/2$, it is
clear that something must be lost: the $N$ values $f_0,\dots,f_{N-1}$ or
corresponding coefficients $\dfc{f}_0,\dots,\dfc{f}_{N-1}$ cannot contain the
same information as the $2M+1=N+1$ Fourier series coefficients
$\fc{f}_{-M},\dots,\fc{f}_{M}$.  If we insist that the function $f$ match the
specified values, i.e., require that $f(x_j) = f_j$ for $j=0,\dots,N-1$, then
we find that \eqref{eq:FS:j} with $N=2M$ reduces to
\begin{equation}
   f_j = \sum_{k=-M+1}^{M-1} \fc{f}_k e^{2\pi ijk/N} 
       + (-1)^j\left(\fc{f}_{-M} + \fc{f}_{M}\right),
\qquad j=0,\dots,N-1 .
\label{eq:FS:j:expanded}
\end{equation}
If we set $\fc{f}_k=\dfc{f}_k$ for $\abs{k}<M=N/2$ and choose $\fc{f}_{-M}$
and $\fc{f}_{M}$ satisfying $\fc{f}_{-M}+\fc{f}_{M}=\dfc{f}_{M}$, then
\eqref{eq:FS:j:expanded} reduces to the IDFT \eqref{eq:IDFT}, which implies
that the function $f$ defined by \eqref{eq:FS:truncated} with these Fourier
series coefficients $\fc{f}_k$ matches the specified values $f_j$.  It almost
matches the DFT coefficients $\dfc{f}_k$ as well, doing so for $\abs{k}<M$;
however, only the sum $\fc{f}_{-M} + \fc{f}_{M}$ is determined by the
data.  Thus, \emph{there is a one-parameter family of band-limited functions}
$f$ which match the values $f_0,\dots,f_{N-1}$ when $N$ is even, one function
corresponding to each choice of the difference $d:=\fc{f}_{M} - \fc{f}_{-M}$.
Which one is ``right'' or ``best''?

A simple approach is to set $d=\dfc{f}_{M}$, which gives $\fc{f}_{M} =
\dfc{f}_{M}$ and $\fc{f}_{-M} = 0$ so that $\fc{f}_k=\dfc{f}_k$ for
$k=-M+1,\dots,M$.  With this choice the numbers $\dfc{f}_0$, \dots,
$\dfc{f}_{N-1}$ correspond to the function
\begin{equation}
   f(x) = f_1(x) := \sum_{k=-M+1}^{M} \dfc{f}_k e^{2\pi ikx/\period} .
\label{eq:FS:1}
\end{equation}
This function agrees with the data values $f_0,\dots,N-1$ and its Fourier
series coefficients $\fc{f}_k$ match the DFT coefficients $\dfc{f}_k$
(modulo~$N$).  However, this simple approach is problematic, at least in the
case where $f$ is real.\footnote{Even if $f$ is not necessarily real, the
truncation is ``unbalanced'' in $k$ (i.e., $\fc{f}_{-M}$ must vanish but
$\fc{f}_{M}$ may not), which seems unreasonable.  It is not clear whether this
is really a problem.}  Specifically, if the data values $f_j$ are real then we
expect $f$ to be a real function, but the condition $\fc{f}_{-M} =
\CC{\fc{f}}_{M}$ implies we must have $\dfc{f}_{M}=0$---which is not
necessarily what the data gives.  Therefore, interpreting the (real) data
$f_0,\dots,f_{N-1}$ as corresponding to the function \eqref{eq:FS:1} is
contradictory in general:  unless $\dfc{f}_{M}$ happens to be zero, then the
function is not strictly real (i.e., real for all $x$), and if it is forced to
be real by setting $\dfc{f}_{M}=0$, then it no longer matches the values
$f_j$.  

A potentially better choice is to set $d=0$, which gives $\fc{f}_{M} =
\fc{f}_{-M} = \frac12\dfc{f}_{M}$ and corresponds to the function
\begin{equation}
   f(x) = f_2(x) := \sum_{k=-M+1}^{M-1} \dfc{f}_k e^{2\pi ikx/\period}
        + \dfc{f}_{M}\cos\left(\frac{2\pi Mx}{\period}\right) .
\label{eq:FS:2}
\end{equation}
Now in the case where the data $f_0,\dots,f_{N-1}$ is real, the condition
$\fc{f}_{-M} = \CC{\fc{f}}_{M}$ implies only that $\dfc{f}_{M}$ must be
real---which it is automatically, as a result of the DFT as noted above.
Thus, interpreting the data as corresponding to the function \eqref{eq:FS:2}
is at least consistent:  $f$~matches the function values and is real if they
are real.  However, with this interpretation the DFT coefficients $\dfc{f}_k$
do not (quite) match the Fourier series coefficients $\fc{f}_k$:  the factor
of two difference for the highest mode must be taken into account if the
series is to be evaluated on a grid with a different number of points, as in a
Fourier psuedospectral multigrid method~\cite{BrandtEtAl84}.

\pagebreak[1]
\subsubsection{Derivatives of Fourier representations}

If $f$ is any function given by a truncated Fourier series of the form
\eqref{eq:FS:truncated}, then its derivative $g=f'$ is given exactly by the
truncated Fourier series
\begin{equation}
   g(x) = \sum_{k=-M}^{M} \fc{g}_k e^{2\pi ikx/\period},
\qquad x\in\R,
\label{eq:FS:truncated:deriv}
\end{equation}
where
\begin{equation}
   \fc{g}_k = \left(\frac{2\pi i k}{\period}\right) \fc{f}_k ,
\qquad k=-M,\dots,M.
\label{eq:FC:deriv}
\end{equation}
Thus, the fact that the Fourier basis functions $\exp(2\pi ikx/\period)$ are
eigenfunctions of the derivative operator makes the representation of this
operator ``diagonal'' in terms of the Fourier series coefficients $\fc{f}_k$.
Is this same result true for the DFT coefficients $\dfc{f}_k$?  That is, can
we replace \eqref{eq:FC:deriv} by
\begin{equation}
   \dfc{g}_k = \left(\frac{2\pi i k}{\period}\right) \dfc{f}_k 
\label{eq:DFC:deriv}
\end{equation}
for the appropriate values of $k$?  

If we represent the truncated series \eqref{eq:FS:truncated} using $N>2M$
values $f_j$ or coefficients $\dfc{f}_k$ related by the DFT pair, the answer
is yes, since the DFT coefficients $\dfc{f}_k$ match the Fourier series
coefficients $\fc{f}_k$ exactly.  More precisely, just as evaluating the
truncated series \eqref{eq:FS:truncated} at $x_j=j\period/N$ gives the IDFT
\eqref{eq:IDFT} with $\dfc{f}_k=\fc{f}_k$ for $k=0,\dots,N-1$ (with
$\fc{f}_k=0$ for $\abs{k}>M$, and $\dfc{f}_k$ is interpreted modulo $N$),
evaluating the truncated series \eqref{eq:FS:truncated:deriv} at the same
points gives
\begin{equation}
   g_j = g\left(\frac{j\period}{N}\right) 
       = \sum_{k=0}^{N-1} \dfc{g}_k e^{2\pi ijk/N},
\qquad j=0,\dots,N-1 ,
\label{eq:IDFT:g}
\end{equation}
where $\dfc{g}_k=\fc{g}_k$ and \eqref{eq:DFC:deriv} holds for $k=0,\dots,N-1$
(interpreted modulo $N$).  Conversely, given either of the $N$-periodic
sequences $f_0,\dots,f_{N-1}$ or $\dfc{f}_0$, \dots, $\dfc{f}_{N-1}$ related
by the DFT pair, if $N$ is \emph{odd} there is a unique function $f$ of the
form \eqref{eq:FS:truncated} with $M=(N-1)/2$ satisfying $f(x_j)=f_j$, and its
derivative \eqref{eq:FS:truncated:deriv} will have the values
\eqref{eq:IDFT:g} as above.  

However, if $N$ is \emph{even} and we attempt to use either of the sequences
$f_0,\dots,f_{N-1}$ or $\dfc{f}_0,\dots,\dfc{f}_{N-1}$ (which are related by
the DFT pair) to determine a band-limited function $f$ and then find its
derivative, the situation is more complicated.  As discussed above, we will
choose $f$ of the form \eqref{eq:FS:truncated} with $M=N/2$ and coefficients
$\fc{f}_k=\dfc{f}_k$ for $\abs{k}<M$; however, the coefficients with
$\abs{k}=M$ need only satisfy $\fc{f}_{-M}+\fc{f}_{M}=\dfc{f}_{M}$.
If we choose $\fc{f}_{-M}=0$ and $\fc{f}_M=\dfc{f}_M$ so that $f$ is given by
\eqref{eq:FS:1}, its derivative is 
\begin{equation}
   g(x) = f_1'(x) = \sum_{k=-M+1}^{M} \dfc{g}_k e^{2\pi ikx/\period} 
\label{eq:FS:1:deriv}
\end{equation}
with $\dfc{g}_k$ given by \eqref{eq:DFC:deriv}.  However, it must be recalled
that for real data $f_0,\dots,f_{N-1}$ the function $f_1(x)$ is real for all
$x$ only if the highest Fourier mode is dropped.  In particular, evaluating
\eqref{eq:FS:1:deriv} at $x_j=j\period/N$ and using \eqref{eq:DFC:deriv} gives
\begin{equation}
   g_j = g(x_j) = \sum_{k=-M+1}^{M-1} \dfc{g}_k e^{2\pi ijk/N} 
       + (-1)^j\left(\frac{2\pi iM}{\period}\right)\dfc{f}_{M},
\qquad j=0,\dots,N-1 ,
\label{eq:FS:deriv:1:expanded}
\end{equation}
and since $\dfc{f}_{M}$ is real, this results in \emph{complex} values of the
derivative.\footnote{If \eqref{eq:FS:deriv:1:expanded} is evaluated using an
FFT code for real functions, the imaginary part will probably be dropped.}
Once again, we conclude that this interpretation of the data is problematic.

On the other hand, if we choose $\fc{f}_{M} = \fc{f}_{-M} =
\frac12\dfc{f}_{M}$ so that $f$ is given by \eqref{eq:FS:2}, its derivative is
\begin{equation}
   g(x) = f_2'(x) = \sum_{k=-M+1}^{M-1} \dfc{g}_k e^{2\pi ikx/\period}
        - \left(\frac{2\pi M}{\period}\right) 
          \dfc{f}_{M}\sin\left(\frac{2\pi Mx}{\period}\right) ,
\label{eq:FS:2:deriv}
\end{equation}
with $\dfc{g}_k$ given by \eqref{eq:DFC:deriv} for $\abs{k}<M$.  It should
be noted that when evaluated at $x_j=j\period/N$ the last term in
\eqref{eq:FS:2:deriv} vanishes, so the contribution from the highest Fourier
mode is lost at those points.  This problem can be solved and exact values of
the derivative obtained by evaluating the at the \emph{midpoints} $x_{j+1/2} =
(j+\tfrac12)\period/N$ instead; we will not pursue this approach here (for
details, see~\cite{BrandtEtAl84}).

\pagebreak[1]
\subsubsection{Transforms of nonlinear terms}

Suppose the model contains a quadratic nonlinear term $f(x)=g(x)h(x)$, 
where $g$ and $h$ are represented in the model by truncated Fourier series
\begin{equation}
   g(x) = \sum_{k=-M}^{M} \fc{g}_k e^{2\pi ikx/\period},
\qquad
   h(x) = \sum_{k=-M}^{M} \fc{h}_k e^{2\pi ikx/\period},
\qquad x\in\R.
\label{eq:FS:g,h}
\end{equation}
By direct multiplication we have
\begin{equation}
   f(x) = \sum_{k=-2M}^{2M} \fc{f}_k e^{2\pi ikx/\period},
\qquad x\in\R.
\label{eq:FS:g*h}
\end{equation}
where the coefficients are given explicitly by
\begin{equation}
   \fc{f}_k = \frac{1}{\period}\int_{0}^{\period} f(x) 
              e^{-2\pi ikx/\period} \,dx 
            = \sum_{l=-M}^{M} \fc{g}_l \fc{h}_{k-l}
\qquad k\in\Z,
\label{eq:FC:g*h}
\end{equation}
where $\fc{g}_k$ and $\fc{h}_k$ vanish for $\abs{k}>M$.  While $f$ contains
contributions from Fourier modes $\abs{k}\le 2M$, we assume that only the
modes $\abs{k}\le M$ will be retained in the model, so we want to compute the
Fourier coefficients $\fc{f}_k$ for $k=-M,\dots,M$ \emph{without aliasing},
i.e., compute these coefficients exactly and simply drop the higher
coefficients.

The formula on the right side of \eqref{eq:FC:g*h} shows how the Fourier modes
in $g$ and $h$ interact to produce Fourier mode~$k$ in~$f$.  However, it is
computationally too expensive to evaluate this sum directly, since it requires
$O(M)$ operations for each~$k$ and thus $O(M^2)$ operations to compute
$\fc{f}_k$ for $k=-M,\dots,M$.  Instead, we use the \emph{transform method}
(e.g.,~\cite{Orszag70}) as follows:
\begin{enumerate}
\item From the Fourier coefficients $\fc{g}_k$ and $\fc{h}_k$ for 
$k=-M,\dots,M$, evaluate $g$ and $h$ on the transform grid $x_j=j\period/N$
for $j=0,\dots,N-1$ in physical space, i.e., compute $g_j=g(x_j)$ and
$h_j=h(x_j)$.
\item Multiply these values together to evaluate $f$ on the transform grid,
i.e., $f_j=g_j h_j$ for $j=0,\dots,N-1$.
\item From the values $f_j$ compute the Fourier coefficients 
$\fc{f}_k$ for $k=-M,\dots,M$.
\end{enumerate}
The key to the efficiency of the method is the fact that we can compute the
transforms in steps~1 and~3 exactly by an IDFT and DFT, respectively, if the
length~$N$ is large enough.  Since step~2 requires $N$ operations, if we use
the FFT algorithm then the overall operation count is only
$O\big(N\log(N)\big)$ operations.

To show that the transforms can be computed exactly---and to establish the
transform length required---we use the following:
\begin{lemma}\label{lemma:integration}
The left-sum approximation
\begin{equation}
   I(\phi) :=  \frac{1}{\period }\int_0^\period \phi(x)\,dx 
          \approx \frac{1}{N}\sum_{j=0}^{N-1}
          \phi\left(\frac{j\period}{N}\right) =: I_N(\phi)
\label{eq:lemma}
\end{equation}
is exact on the function $\phi_k(x) = \exp(2\pi ikx/\period)$ with $k\in\Z$
provided that $\abs{k}<N$.  
%Furthermore, if $\abs{k}=N$ then $I_N(\phi_k) = 1$.
\end{lemma}
\begin{proof}
The true integral is
\begin{equation}
   I(\phi_k) =  \frac{1}{\period}\int_0^\period e^{2\pi ikx/\period} \,dx 
             = \begin{cases} 1, & \text{$k=0$},\\
                             0, & \text{$k\ne0$}, \end{cases}
\label{eq:I}
\end{equation}
and by Lemma~\ref{lemma:orthogonality} the left-sum approximation is
\begin{equation}
   I_N(\phi_k) = \frac{1}{N}\sum_{j=0}^{N-1} e^{2\pi ijk/N} 
               = \begin{cases} 1, & \text{$k\equiv0\pmod{N}$},\\
                               0, & \text{otherwise}. \end{cases}
\label{eq:I_N}
\end{equation}
Comparing \eqref{eq:I} and \eqref{eq:I_N} we see that $I(\phi_k)=I_N(\phi_k)$
provided that $\abs{k}<N$.  
%Furthermore, \eqref{eq:I_N} also shows that if $\abs{k}=N$ 
%then $I_N(\phi_k)=1$.
\end{proof}

To apply this Lemma we express the left side of \eqref{eq:FC:g*h} in the 
notation introduced above and substitute from \eqref{eq:FS:g*h} with $k$ 
replaced by $l$ to obtain
\begin{equation}
   \fc{f}_k = I\left(f\phi_{-k}\right) 
            = I\left(\sum_{l=-2M}^{2M} \fc{f}_l \phi_l \phi_{-k}\right)
            = \sum_{l=-2M}^{2M} \fc{f}_l I\left(\phi_{l-k}\right),
\label{eq:FC:g*h:I}
\end{equation}
which we want to compute exactly for $\abs{k}\le M$.  Likewise, the left-sum
approximation using $N$ points is
\begin{equation}
   \frac{1}{N}\sum_{j=0}^{N-1} f_j e^{-2\pi ijk/N} = I_N\left(f\phi_{-k}\right) 
      = \sum_{l=-2M}^{2M} \fc{f}_l I_N\left(\phi_{l-k}\right).
\label{eq:FC:g*h:I_N}
\end{equation}
Since $\abs{l}\le 2M$ and $\abs{k}\le n$, we have $\abs{k-l}\le 3M$, so by 
Lemma~\ref{lemma:integration}, $I_N(\phi_{k-l})=I(\phi_{k-l})$ and thus
\begin{equation}
   \fc{f}_k = \frac{1}{N}\sum_{j=0}^{N-1} f_j e^{2\pi ijk/N},
\qquad k=-M,\dots,M
\end{equation}
provided that $N>3M$.  Therefore, the transform method as described above
computes the coefficients of the quadratic nonlinear term $f=gh$ exactly if
$N>3M$, where $N$ is the transform length and $M$ is the spectral truncation,
i.e., the index of the highest Fourier coefficient retained in the
spectral representations of $f$, $g$, and $h$.  Since the transform length $N$
is greater than the number $2M+1$ of Fourier coefficients retained, the
coefficients of $g$ and $h$ must be ``padded'' with zeros before computing the
IDFT (see Fig.~\ref{fig:coefficients} for an example of this padding).
Likewise, the coefficients of $f$ computed by the DFT must be truncated, i.e.,
the coeffients $\fc{f}_k$ for $\abs{k}>M$ must be set to zero.\footnote{It is
possible that by interpreting the DFT coefficients in terms of ``unbalanced''
discrete Fourier representations such as \eqref{eq:FS:1} the transform length
$N$ needed for exact results may be slightly smaller; we will not pursue this
possiblity here.}

\pagebreak[4]
\subsection{Fourier representations in two dimensions}

The model domain for \pswm\ is doubly periodic with period $\period_x$ and
$\period_y$ in $x$ and $y$ respectively.  Here we give the details of
two-dimensional Fourier representations on this domain.

\subsubsection{Discrete Fourier transforms}

Suppose that a given function $f$ can be represented as a truncated double
Fourier series with spectral truncation $M_x$ and $M_y$ in $x$ and $y$,
respectively.  Repeating the one-dimensional representation developed above
gives
\begin{equation}
   f(x,y) = \sum_{l=-M_x}^{M_x} \fc{f}_{l}(y) e^{2\pi ilx/\period_x}
          = \sum_{l=-M_x}^{M_x} \sum_{m=-M_y}^{M_y}
      \fc{f}_{l,m} e^{2\pi ilx/\period_x} e^{2\pi imy/\period_y},
\qquad x,y\in\R,
\label{eq:FS2}
\end{equation}
where
\begin{equation}
   \fc{f}_l(y) = \frac{1}{\period_x}\int_{0}^{\period_x} f(x,y) 
              e^{-2\pi ilx/\period_x} \,dx ,
\quad
   \fc{f}_{l,m} = \frac{1}{\period_y}\int_{0}^{\period_y} \fc{f}_l(y) 
              e^{-2\pi imy/\period_y} \,dy ,
\qquad l,m\in\Z.
\label{eq:FC2}
\end{equation}
Evaluating the series in \eqref{eq:FS2} at $x_j=j\period_x/N_x$ and
$y_k=k\period_y/N_y$ gives
\begin{equation}
   f_{j,k} = f(x_j,y_k) = \sum_{l=-M_x}^{M_x} \fc{f}_{l}(y_k) e^{2\pi ijl/N_x}
          = \sum_{l=-M_x}^{M_x} \sum_{m=-M_y}^{M_y}
      \fc{f}_{l,m} e^{2\pi ijl/N_x} e^{2\pi ikm/N_y}
\label{eq:FS2:discrete}
\end{equation}
for $j=0,\dots,N_x-1$ and $k=0,\dots,N_y-1$.
As explained above, if $N_x>2M_x$ and $N_y>2M_y$, the Fourier series
coefficients \eqref{eq:FC2} may be evaluated exactly by the DFTs
\begin{equation}
   \Fc{f}_l(y_k) = \frac{1}{N_x}\sum_{j=0}^{N_x-1} f_{j,k}
              e^{-2\pi ijl/N_x} ,
\quad
   \Fc{f}_{l,m} = \frac{1}{N_y}\sum_{k=0}^{N_y-1} \Fc{f}_l(y_k) 
              e^{-2\pi ikm/N_y} 
\label{eq:DFT2}
\end{equation}
for $l=0,\dots,N_x-1$ and $m=0,\dots,N_y-1$; the DFT coefficients
$\Fc{f}_l(y_k)$ and $\Fc{f}_{l,m}$ match the Fourier series coefficients
$\fc{f}_l(y_k)$ and $\fc{f}_{l,m}$ for $\abs{l}\le M_x$ and $\abs{m}\le M_y$
and vanish for larger $l$ and $m$.  Since the DFT coeffcients are periodic, we
can write the series \eqref{eq:FS2:discrete} using IDFTs, i.e., in the form
\begin{equation}
   f_{j,k} = \sum_{l=0}^{N_x-1} \Fc{f}_{l}(y_k) e^{2\pi ijl/N_x}
          = \sum_{l=0}^{N_x-1} \sum_{m=0}^{N_y-1}
      \Fc{f}_{l,m} e^{2\pi ijl/N_x} e^{2\pi ikm/N_y}
\label{eq:IDFT2}
\end{equation}
for $j=0,\dots,N_x-1$ and $k=0,\dots,N_y-1$.  It should be noted that if
$N_x=2M_x$ and $N_y=2M_y$ then the above results still hold, except that
$\fc{f}_l(y_k) = \frac12\Fc{f}_l(y_k)$ for $\abs{l}=M_x$ and $\fc{f}_{l,m} =
\frac12\Fc{f}_{l,m}$ for $\abs{m}=M_y$.  Also, if $f$ is real (the only case
in which we are interested) then
\begin{equation}
   \fc{f}_{-l}(y) = \CC{\fc{f}}_l(y),
\qquad
   \fc{f}_{-l,-m} = \CC{\fc{f}}_{l,m},
\label{eq:CC}
\end{equation}
and
\begin{equation}
   \Fc{f}_{-l}(y_k) = \CC{\Fc{f}}_l(y_k).
\qquad
   \Fc{f}_{-l,-m} = \CC{\Fc{f}}_{l,m}.
\label{eq:CC:discrete}
\end{equation}

\subsubsection{Computation and storage}

On the computer, we use the FFT routine \code{fft99} by Temperton
\cite{Temperton83a,Temperton83b,Temperton83c} to compute the DFTs and IDFTs in
\eqref{eq:DFT2} and \eqref{eq:IDFT2}, respectively.  This routine computes
real-to-``half-complex'' transforms (or vice-versa), so applying it to the
$y$-transforms---which are complex-to-complex transforms since $\Fc{f}_l(y_k)$
are complex numbers---does not produce the numbers $\Fc{f}_{k,l}$ directly.
The procedure for using this routine is explained below and illustrated in 
Fig.~\ref{fig:transforms}.

We start with the numbers $f_{j,k}$ stored with explicit periodicity, i.e.,
stored for $j=-1,\dots,N_x$ and $k=-1,\dots,N_y$ as shown in
Fig.~\ref{fig:transforms} (top part).  ``Explicit periodicity'' means that
$f_{-1,k}=f_{N_x-1,k}$ and $f_{Nx,k}=f_{0,k}$ (with analogous periodicity
in~$k$); the two ``extra'' storage locations are needed for the calculations.
The $x$-transform [first part of \eqref{eq:DFT2}] produces the half-complex
sequence $\Fc{f}_{l}(y_k)$ for each $k=-1,\dots,N_y$; what is actually
returned by the routine \code{fft99} consists of the real and imaginary parts
of this sequence, i.e., the real numbers $a_{l,k}$ and $b_{l,k}$ satisfying
\begin{equation} 
   \Fc{f}_{l}(y_k) = a_{l,k} + ib_{l,k} 
\label{eq:FC2:RI}
\end{equation} 
for $l=0,\dots,M_x$, stored as shown in Fig.~\ref{fig:transforms} (middle
part).  Here we use the notation $M_x=N_x/2$ for simplicity.\footnote{This is
just notation here, \emph{not} the spectral truncation.  The model actually
uses the spectral trunctation $M_x<N_x/3$ in order to compute quadratic
nonlinearities without aliasing as discussed previously.}  The remaining
coefficients $\Fc{f}_{l}(y_k)$ for $l=Mx+1,\dots,Nx-1$ are not computed
explicitly but could be computed from $\Fc{f}_{Nx-l}(y_k) =
\CC{\Fc{f}}_{l}(y_k) = a_{l,k} - ib_{l,k}$.  Also, $b_{0,k}=b_{M_x,k}=0$
(since the values $f_{j,k}$ are real) corresponding to the two ``extra''
storage locations used for the input values.

Likewise, for the $y$ transform [second part of \eqref{eq:DFT2}],
since the DFT is linear we have
\begin{equation}
   \Fc{f}_{l,m} = \Fc{a}_{l,m} + i\Fc{b}_{l,m}
\label{eq:DFT2:RI}
\end{equation}
where
\begin{equation}
   \Fc{a}_{l,m} = \frac{1}{N_y}\sum_{k=0}^{N_y-1} a_{l,k} e^{-2\pi ikm/N_y},
\qquad
   \Fc{b}_{l,m} = \frac{1}{N_y}\sum_{k=0}^{N_y-1} b_{l,k} e^{-2\pi ikm/N_y}
\label{eq:DFT2:a,b}
\end{equation}
are the DFTs of the (real) sequences $a_{l,k}$ and $b_{l,k}$, respectively.
The sequences $\Fc{a}_{l,m}$ and $\Fc{b}_{l,m}$ are half-complex, i.e.,
\begin{equation}
   \Fc{a}_{l,N_y-m}=\CC{\Fc{a}}_{l,m},
\qquad
   \Fc{b}_{l,N_y-m}=\CC{\Fc{b}}_{l,m},
\label{eq:CC:a,b}
\end{equation}
so $\code{fft99}$ returns their real and imaginary parts, which we denote here 
by
\begin{equation}
   \Fc{a}_{l,m} = A_{l,m} + iC_{l,m},
\qquad
   \Fc{b}_{l,m} = B_{l,m} + iD_{l,m}
\label{eq:DFT2:RI:A,B}
\end{equation}
for $m=0,\dots,M_y=N_y/2$ as shown in Fig.~\ref{fig:transforms} (bottom
part).  Once again, since the sequences $a_{l,k}$ and $b_{l,k}$ are real, 
we have ${C}_{l,0}={C}_{l,M_y}=0$ and ${D}_{l,0}={D}_{l,M_y}=0$.

\renewcommand{\ss}[1]{{\scriptstyle #1}}
\begin{figure}
\begin{displaymath}
\begin{matrix}
\ss{k=N_y}  & f_{-1,N_y} & f_{0,N_y} & f_{1,N_y} & f_{2,N_y} & \dots &
f_{j,N_y} & \dots & f_{N_x-1,N_y} & f_{N_x,N_y} 
\\[4pt]
\ss{k=N_y-1}  & f_{-1,N_y-1} & f_{0,N_y-1} & f_{1,N_y-1} & f_{2,N_y-1} & \dots &
f_{j,N_y-1} & \dots & f_{N_x-1,N_y-1} & f_{N_x,N_y-1} 
\\[4pt]
\vdots  & \vdots & \vdots & \vdots & \vdots & & \vdots & 
& \vdots & \vdots 
\\[4pt]
\ss{k}  & f_{-1,k} & f_{0,k} & f_{1,k} & f_{2,k} & \dots & f_{j,k} & \dots
& f_{N_x-1,k} & f_{N_x,k} 
\\[4pt]
\vdots  & \vdots & \vdots & \vdots & \vdots & & \vdots & 
& \vdots & \vdots 
\\[4pt]
%\ss{k=2}  & f_{-1,2} & f_{0,2} & f_{1,2} & f_{2,2} & \dots & f_{j,2} & \dots
%& f_{N_x-1,2} & f_{N_x,2} 
%\\[4pt]
%\ss{k=1}  & f_{-1,1} & f_{0,1} & f_{1,1} & f_{2,1} & \dots & f_{j,1} & \dots
%& f_{N_x-1,1} & f_{N_x,1} 
%\\[4pt]
\ss{k=0}  & f_{-1,0} & f_{0,0} & f_{1,0} & f_{2,0} & \dots & f_{j,0} & \dots
& f_{N_x-1,0} & f_{N_x,0} 
\\[4pt]
\ss{k=-1}  & f_{-1,-1} & f_{0,-1} & f_{1,-1} & f_{2,-1} & \dots & f_{j,-1} & \dots
& f_{N_x-1,-1} & f_{N_x,-1} 
\\[6pt]
   & \ss{j=-1} & \ss{j=0} & \ss{j=1} & \ss{j=2} & \dots & \ss{j} & \dots &
\ss{j=N_x-1} & \ss{j=N_x} 
\end{matrix}
\end{displaymath}
\begin{displaymath}
   \Downarrow\text{forward $x$-transform} \Downarrow 
   \qquad \qquad 
   \Uparrow \text{inverse $x$-transform} \Uparrow
\end{displaymath}
\begin{displaymath}
\setcounter{MaxMatrixCols}{12}
\begin{matrix}
\ss{k=N_y}  & a_{0,N_y} & b_{0,N_y} & a_{1,N_y} & b_{1,N_y} & \dots &
a_{l,N_y} & b_{l,N_y} & \dots & a_{M_x,N_y} & b_{M_x,N_y} 
\\[4pt]
\ss{k=N_y-1}  & a_{0,N_y-1} & b_{0,N_y-1} & a_{1,N_y-1} & b_{1,N_y-1} & \dots &
a_{l,N_y-1} & b_{l,N_y-1} & \dots & a_{M_x,N_y-1} & b_{M_x,N_y-1} 
\\[4pt]
\vdots  & \vdots & \vdots & \vdots & \vdots & & \vdots & 
& \vdots & \vdots 
\\[4pt]
\ss{k}  & a_{0,k} & b_{0,k} & a_{1,k} & b_{1,k} & \dots &
a_{l,k} & b_{l,k} & \dots & a_{M_x,k} & b_{M_x,k} 
\\[4pt]
\vdots  & \vdots & \vdots & \vdots & \vdots & & \vdots & 
& \vdots & \vdots 
\\[4pt]
%\ss{k=2}  & a_{0,2} & b_{0,2} & a_{1,2} & b_{1,2} & \dots &
%a_{l,2} & b_{l,2} & \dots & a_{M_x,2} & b_{M_x,2} 
%\\[4pt]
%\ss{k=1}  & a_{0,1} & b_{0,1} & a_{1,1} & b_{1,1} & \dots &
%a_{l,1} & b_{l,1} & \dots & a_{M_x,1} & b_{M_x,1} 
%\\[4pt]
\ss{k=0}  & a_{0,0} & b_{0,0} & a_{1,0} & b_{1,0} & \dots &
a_{l,0} & b_{l,0} & \dots & a_{M_x,0} & b_{M_x,0} 
\\[4pt]
\ss{k=-1}  & a_{0,-1} & b_{0,-1} & a_{1,-1} & b_{1,-1} & \dots &
a_{l,-1} & b_{l,-1} & \dots & a_{M_x,-1} & b_{M_x,-1} 
\\[6pt]
   & \ss{j=-1} & \ss{j=0} & \ss{j=1} & \ss{j=2} & \dots & \ss{j=2l-1} &
\ss{j=2l} & \dots & \ss{j=N_x-1} & \ss{j=N_x} 
%   & \multicolumn{2}{c}{l=0} & \multicolumn{2}{c}{l=1} & \dots 
%   & \multicolumn{2}{c}{l} & \dots & \multicolumn{2}{c}{l=M_x} 
\end{matrix}
\end{displaymath}
\begin{displaymath}
   \Downarrow\text{forward $y$-transform} \Downarrow 
   \qquad \qquad 
   \Uparrow \text{inverse $y$-transform} \Uparrow
\end{displaymath}
\begin{displaymath}
\setcounter{MaxMatrixCols}{12}
\begin{matrix}
\ss{k=N_y}  & C_{0,M_y} & D_{0,M_y} & C_{1,M_y} & D_{1,M_y} & \dots &
C_{l,M_y} & D_{l,M_y} & \dots & C_{M_x,M_y} & D_{M_x,M_y} 
\\[4pt]
\ss{k=N_y}  & A_{0,M_y} & B_{0,M_y} & A_{1,M_y} & B_{1,M_y} & \dots &
A_{l,M_y} & B_{l,M_y} & \dots & A_{M_x,M_y} & B_{M_x,M_y} 
\\[4pt]
\vdots  & \vdots & \vdots & \vdots & \vdots & & \vdots & 
& \vdots & \vdots 
\\[4pt]
\ss{k=2m}  & C_{0,m} & D_{0,m} & C_{1,m} & D_{1,m} & \dots &
C_{l,m} & D_{l,m} & \dots & C_{M_x,m} & D_{M_x,m} 
\\[4pt]
\ss{k=2m-1}  & A_{0,m} & B_{0,m} & A_{1,m} & B_{1,m} & \dots &
A_{l,m} & B_{l,m} & \dots & A_{M_x,m} & B_{M_x,m} 
\\[4pt]
\vdots  & \vdots & \vdots & \vdots & \vdots & & \vdots & 
& \vdots & \vdots 
\\[4pt]
\ss{k=2}  & C_{0,1} & D_{0,1} & C_{1,1} & D_{1,1} & \dots &
C_{l,1} & D_{l,1} & \dots & C_{M_x,1} & D_{M_x,1} 
\\[4pt]
\ss{k=1}  & A_{0,1} & B_{0,1} & A_{1,1} & B_{1,1} & \dots &
A_{l,1} & B_{l,1} & \dots & A_{M_x,1} & B_{M_x,1} 
\\[4pt]
\ss{k=0}  & C_{0,0} & D_{0,0} & C_{1,0} & D_{1,0} & \dots &
C_{l,0} & D_{l,0} & \dots & C_{M_x,0} & D_{M_x,0} 
\\[4pt]
\ss{k=-1}  & A_{0,0} & B_{0,0} & A_{1,0} & B_{1,0} & \dots &
A_{l,0} & B_{l,0} & \dots & A_{M_x,0} & B_{M_x,0} 
\\[6pt]
   & \ss{j=-1} & \ss{j=0} & \ss{j=1} & \ss{j=2} & \dots & \ss{j=2l-1} &
\ss{j=2l} & \dots & \ss{j=N_x-1} & \ss{j=N_x} 
\end{matrix}
\end{displaymath}
\begin{center}
\mycaption[4.5in]{\label{fig:transforms}
Storage for values of a function $f$ and corresponding Fourier coefficients
(here $M_x=N_x/2$ and $M_y=N_y/2$).}
\end{center}
\vspace*{-2pt}
\end{figure}

\pagebreak[2]

From \eqref{eq:DFT2:RI} and \eqref{eq:DFT2:RI:A,B} we see that for each
$l=0,\dots,N_x$ and $m=0,\dots,N_y$ the four numbers $A_{l,m}$, $B_{l,m}$,
$C_{l,m}$, and $D_{l,m}$ yield the Fourier coefficients
\begin{align}
   \Fc{f}_{l,m} &= (A_{l,m} - D_{l,m}) + i(B_{l,m} + C_{l,m}) ,
\label{eq:DFT2:RI:results:a}
\\
\intertext{and using \eqref{eq:CC:discrete} and \eqref{eq:CC:a,b} we find that}
   \Fc{f}_{Nx-l,m} &= (A_{l,m} + D_{l,m}) - i(B_{l,m} - C_{l,m}) ,
\\
   \Fc{f}_{l,N_y-m} &= (A_{l,m} + D_{l,m}) + i(B_{l,m} - C_{l,m}) ,
\\
   \Fc{f}_{N_x-l,N_y-m} &= (A_{l,m} - D_{l,m}) - i(B_{l,m} + C_{l,m}) .
\label{eq:DFT2:RI:results}
\end{align}
We can denote these formulas by the relation
\begin{equation}
   \Fc{f}_{l,m} \leftrightarrow
   \begin{bmatrix} C_{l,m} & D_{l,m} \\ A_{l,m} & B_{l,m} \end{bmatrix}
\label{eq:DFC:matrix}
\end{equation}
where the array on the right-hand side is neither a matrix nor a stencil, but
simply a schematic representation of the locations of the four numbers
$A_{l,m}$, $B_{l,m}$, $C_{l,m}$, and $D_{l,m}$ related to the coefficient
$\Fc{f}_{l,m}$ in the array produced by two applications of the FFT routine
\code{fft99}.
However, it is probably never necessary to actually compute the Fourier
coefficients $\Fc{f}_{l,m}$ explicitly; all necessary calculations can be done
with these four numbers directly.  For testing the transforms, we can invert
\eqref{eq:DFT2:RI:results:a}--\eqref{eq:DFT2:RI:results} to obtain
\begin{align}
   A_{l,m} &= \frac{1}{4}\left(
      \Fc{f}_{l,m} + \Fc{f}_{-l,m} + \Fc{f}_{l,-m} + \Fc{f}_{-l,-m} \right) ,
\\
   B_{l,m} &= \frac{1}{4i}\left(
      \Fc{f}_{l,m} - \Fc{f}_{-l,m} + \Fc{f}_{l,-m} - \Fc{f}_{-l,-m} \right) ,
\\
   C_{l,m} &= \frac{1}{4i}\left(
      \Fc{f}_{l,m} + \Fc{f}_{-l,m} - \Fc{f}_{l,-m} - \Fc{f}_{-l,-m} \right) ,
\\
   D_{l,m} &= \frac{1}{4}\left(
     -\Fc{f}_{l,m} + \Fc{f}_{-l,m} + \Fc{f}_{l,-m} - \Fc{f}_{-l,-m} \right) .
\label{eq:DFT2:inverted}
\end{align}

\subsubsection{Derivatives}

The most important operations computed in spectral space are the derivatives.
Letting $g$ and $h$ denote the $x$ and $y$ derivatives of $f$, respectively,
then by \eqref{eq:DFC:deriv} their spectral coefficients are given by
\begin{equation}
   \dfc{g}_{l,m} = \left(\frac{2\pi i l}{\period_x}\right) \dfc{f}_{l,m} ,
\qquad
   \dfc{h}_{l,m} = \left(\frac{2\pi i m}{\period_y}\right) \dfc{f}_{l,m} .
\label{eq:DFC:deriv:xy}
\end{equation}
Using \eqref{eq:DFT2:RI:results} in these formulas gives
\begin{align}
   \Fc{g}_{l,m} &= \left(\frac{2\pi l}{\period_x}\right)
      \Big[ -(B_{l,m} + C_{l,m}) + i(A_{l,m} - D_{l,m})\Big] ,
\\
   \Fc{g}_{Nx-l,m} &= \left(\frac{2\pi l}{\period_x}\right)
      \Big[ -(B_{l,m} - C_{l,m}) - i(A_{l,m} + D_{l,m})\Big] ,
\\
   \Fc{g}_{l,N_y-m} &= \left(\frac{2\pi l}{\period_x}\right)
      \Big[ -(B_{l,m} - C_{l,m}) + i(A_{l,m} + D_{l,m})\Big] ,
\\
   \Fc{g}_{N_x-l,N_y-m} &= \left(\frac{2\pi l}{\period_x}\right)
      \Big[ -(B_{l,m} + C_{l,m}) - i(A_{l,m} - D_{l,m})\Big] 
\label{eq:DFT2:RI:deriv:x}
\end{align}
and
\begin{align}
   \Fc{h}_{l,m} &= \left(\frac{2\pi m}{\period_y}\right)
      \Big[ -(B_{l,m} + C_{l,m}) + i(A_{l,m} - D_{l,m})\Big] ,
\\
   \Fc{h}_{Nx-l,m} &= \left(\frac{2\pi m}{\period_y}\right)
      \Big[ (B_{l,m} - C_{l,m}) + i(A_{l,m} + D_{l,m})\Big] ,
\\
   \Fc{h}_{l,N_y-m} &= \left(\frac{2\pi m}{\period_y}\right)
      \Big[ (B_{l,m} - C_{l,m}) - i(A_{l,m} + D_{l,m})\Big] ,
\\
   \Fc{h}_{N_x-l,N_y-m} &= \left(\frac{2\pi m}{\period_y}\right)
      \Big[ -(B_{l,m} + C_{l,m}) - i(A_{l,m} - D_{l,m})\Big] .
\label{eq:DFT2:RI:deriv:y}
\end{align}
These may be summarized in the schematic notation of \eqref{eq:DFC:matrix}
in the form
\begin{equation}
   (\Fc{f_{x}})_{l,m} \leftrightarrow\left(\frac{2\pi l}{\period_x}\right)
   \begin{bmatrix} -D_{l,m} & C_{l,m} \\ -B_{l,m} & A_{l,m} \end{bmatrix},
\qquad
   (\Fc{f_{y}})_{l,m} \leftrightarrow\left(\frac{2\pi m}{\period_y}\right)
   \begin{bmatrix} A_{l,m} & B_{l,m} \\ -C_{l,m} & -D_{l,m} \end{bmatrix}.
\label{eq:DFC:matrix:deriv}
\end{equation}
Also, second derivatives have the simpler representation
\begin{equation}
   (\Fc{f_{xx}})_{l,m} \leftrightarrow -\left(\frac{2\pi l}{\period_x}\right)^2
   \begin{bmatrix} A_{l,m} & B_{l,m} \\ C_{l,m} & D_{l,m} \end{bmatrix},
\qquad
   (\Fc{f_{yy}})_{l,m} \leftrightarrow -\left(\frac{2\pi m}{\period_y}\right)^2
   \begin{bmatrix} A_{l,m} & B_{l,m} \\ C_{l,m} & D_{l,m} \end{bmatrix}.
\label{eq:DFC:matrix:deriv2}
\end{equation}
The important point here is that the operations of taking derivatives may be
represented in spectral space by scaling the four numbers corresponding to a
given Fourier $(l,m)$ and rearranging them and changing their signs as needed.  

Also note the point made in the previous section about discrete Fourier
representations when the transform length~$N$ is twice the spectral
truncation~$M$.  Suppose, for example, that we wish to take the $x$-derivative
of a real-valued function $f$ represented in Fourier spectral space with
$N_x=2M_x$.  Since $\Fc{f}_{l}(y_k)$ is real for $l=M_x$ in \eqref{eq:FC2:RI},
we have $b_{M_x,k}=0$ and thus $B_{M_x,m}=D_{M_x,m}=0$ from
\eqref{eq:DFT2:RI:A,B}.  From the first part of \eqref{eq:DFC:matrix:deriv} we
see that the information in Fourier mode $l=M_x$ (i.e., $A_{M_x,m}$ and
$C_{M_x,m}$) will be lost in taking the derivative, since \code{fft99} ignores
the imaginary part of this mode (it must be zero to produce real values).
This problem does not affect the second derivative as in
\eqref{eq:DFC:matrix:deriv2}, nor will it affect derivatives when $M_x<2N_x$.

\pagebreak[3]
\begin{thebibliography}{99}

\bibitem{Fulton07}
Fulton, S. R., 2007:
Semi-Implicit Discretization of the Shallow-Water Equations.
Unpublished manuscript.

\bibitem{BrandtEtAl84}
Brandt, A., S. R. Fulton, and G. D. Taylor, 1984:
Improved spectral multigrid methods for periodic elliptic problems.
\textsl{J.\ Comp.\ Phys.}, \textbf{58}, 96--112.

\bibitem{Orszag70}
Orszag, S. A., 1970:
Transform method for the calculation of vector-coupled sums:
application to the spectral form of the vorticity equation.
\textsl{J. Atmos.\ Sci.}, \textbf{27}, 890--895.

\bibitem{Temperton83a}
Temperton, C., 1983a:  
Self-sorting mixed-radix fast Fourier transforms. 
\textsl{J. Comp.\ Phys.}, \textbf{52}, 1--23.

\bibitem{Temperton83b}
Temperton, C., 1983b:  
A note on prime factor FFT algorithms.  
\textsl{J. Comp.\ Phys.}, \textbf{52}, 198--204.

\bibitem{Temperton83c}
Temperton, C., 1983c:  
Fast mixed-radix real Fourier transforms.  
\textsl{J.  Comp.\ Phys.}, \textbf{52}, 340--350.

\end{thebibliography}

\end{document}
